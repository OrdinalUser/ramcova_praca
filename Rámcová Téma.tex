% Metódy inžinierskej práce - Rámcová téma

\documentclass[10pt,twoside,slovak,a4paper]{article}

\usepackage[slovak]{babel}
\usepackage[IL2]{fontenc} % lepšia sadzba písmena Ľ než v T1
\usepackage[utf8]{inputenc}
\usepackage{graphicx}
\usepackage{url} % príkaz \url na formátovanie URL
\usepackage{hyperref} % odkazy v texte budú aktívne (pri niektorých triedach dokumentov spôsobuje posun textu)
\usepackage{cite}
%\usepackage{times}

\pagestyle{headings}

\title{Regulérne Výrazy\thanks{Semestrálny projekt v predmete Metódy inžinierskej práce, ak. rok 2023/24, vedenie: Ing. Ivan Kapustík}} % meno a priezvisko vyučujúceho na cvičeniach

\author{Tomáš Tytykalo\\[2pt]
	{\small Slovenská technická univerzita v Bratislave}\\
	{\small Fakulta informatiky a informačných technológií}\\
	{\small \texttt{xtytykalo@stuba.sk}}
	}

\date{\small 8. október 2023} % upravte

\begin{document}

\maketitle

\begin{abstract}

Tento článok sa zameriava na aplikáciu regulérnych výrazov v oblasti informatiky a textového spracovania. Regulérne výrazy poskytujú používateľsky jednoduchý a efektívny spôsob vyhľadávania a manipuláciu textových dát. Táto práca začína popisom základov regulérnych výrazov, vrátane ich syntaxe a operácií. Následne sa zameriavame na ich použitie v rôznych kontextoch, vrátane bežného využitia v praxi a teda aj v množstve bežne dostupných programov, alebo programovacích jazykoch. Predstavíme si v ktorých situáciách sa dajú efektívne a jednoducho aplikovať. Regulérne výrazy sú teda mocným nástrojom pre každého dobrého informatika, či programátora.

\end{abstract}

\section{Úvod}

\section{Zápis regulérnych výrazov}
\cite{Sanjiv}
\subsection{Základné regulérne výrazy}
\subsection{Meta znaky}
\subsection{Globálne modifikátory}

\section{Aplikácia}
\cite{Bioperl}
\subsection{Textové editory}
\subsection{Programovacie jazyky}
\subsection{Linuxové prostredie}

\section{Záver}

\bibliography{literatura}
\bibliographystyle{plain} % prípadne alpha, abbrv alebo hociktorý iný

% https://citeseerx.ist.psu.edu/pdf/d06c81bf356235854d83b6bfe29d00eb866926a8

\end{document}