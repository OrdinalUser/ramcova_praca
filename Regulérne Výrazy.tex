% Metódy inžinierskej práce - Rámcová téma

\documentclass[10pt,twoside,slovak,a4paper]{article}

\usepackage[slovak]{babel}
\usepackage[IL2]{fontenc} % lepšia sadzba písmena Ľ než v T1
\usepackage[utf8]{inputenc}
\usepackage{graphicx}
\usepackage{url} % príkaz \url na formátovanie URL
\usepackage{hyperref} % odkazy v texte budú aktívne (pri niektorých triedach dokumentov spôsobuje posun textu)
\usepackage{cite}
%\usepackage{times}
\usepackage{csquotes}
\usepackage{tabularray}
\usepackage{blindtext}
\usepackage{titlesec}
\usepackage{booktabs}

\pagestyle{headings}

\title{Regulérne Výrazy\thanks{Semestrálny projekt v predmete Metódy inžinierskej práce, ak. rok 2023/24, vedenie: Ing. Ivan Kapustík}} % meno a priezvisko vyučujúceho na cvičeniach

\author{Tomáš Tytykalo\\[2pt]
	{\small Slovenská technická univerzita v Bratislave}\\
	{\small Fakulta informatiky a informačných technológií}\\
	{\small \texttt{xtytykalo@stuba.sk}}
	}

\date{\small 8. október 2023} % upravte

\begin{document}

\maketitle

\begin{abstract}

Článok sa zameriava na aplikáciu regulérnych výrazov v oblasti informatiky a textového spracovania. Regulérne výrazy poskytujú používateľsky jednoduchý a efektívny spôsob vyhľadávania a manipuláciu textových dát.
Táto práca začína popisom základov regulérnych výrazov, vrátane ich syntaxe a operácií. Následne sa zameriavame na ich použitie v rôznych kontextoch, vrátane bežného využitia v praxi a teda aj v množstve bežne dostupných programov, alebo programovacích jazykoch.
Predstavíme si v ktorých situáciách sa dajú efektívne a jednoducho aplikovať. Regulérne výrazy sú teda mocným nástrojom pre každého dobrého informatika, či programátora.

\end{abstract}

\clearpage
\tableofcontents
\clearpage

\section{Úvod}
Regulérne výrazy, taktiež známe ako RegEx, či RegExp (Regular Expression) sú populárne a husto podporované v programoch, kde majú zmysel. Ako napríklad textové editory.

Regulérne výrazy predstavujú jednoduchý a efektívny spôsob vyhladávania v textových reťazcoch. Ich aplikácia zahŕňa extrakcia informácií, overovanie uživateľského vstupu, alebo aj nahrádzanie textu. 

Hlavné využitie má v programovacích a skriptovacích ako napríklad JavaScript, PHP, Python ale taktiež aj v kope iných typu C++, C\#. Ich podporu nachádzame bežne v textových editoroch ako nástroj na vyhladávanie a nahrádzanie.

Prejdeme si ich zápis, funkcionalitu, využitie v programoch, a aj v programovacích jazykoch.

Výborný papier na túto tému je \cite{Sanjiv}. Obsahuje podobný obsah, ale s inými príkladmi a iným vysvetlením. V prípade zložitosti tohoto papiera alebo vzniknutých nejasností ho doporučujem skontrolovať.

\section{Zápis regulérnych výrazov}
Pre túto sekciu sme čerpali informácie z publikácie \cite{RegEx_Basics}. Je to podobný papier tejto sekcií s rozdielom že sa aj zaoberá linuxovým prostredím, ktorý sme mi nespomenuli.
Taktiež papier \cite{RegEx_Usage} obsahuje podrobnejšie informácie ohľadom implementácie samotných regulérnych výrazov, ktorá tu tiež nie je spomenutá aby sme nekomplikovali jednoduchosť tohoto papiera.

\subsection{Základné regulérne výrazy}
Jednoduchý regulérny výraz môže vyzerať následovne:
\begin{verbatim}
/Auto/g
\end{verbatim}
Tento výraz nájde všetky prípady 'Auto' v celom textovom reťazci. Poďme si ich ale rozobrať trošku hlbšie. Všetko medzi dvoma // je formát vyhľadávania, teda to čo nás teraz zaujíma. Všetko vonku // je modifikátor, k týmto sa dostaneme neskôr.

\subsubsection{Rozsah}
Výrazy typu hladania jedného reťazca sú pomerne jednoduché. Napíšeme iba vyhľadávaný reťazec, tak ako v predošlom príklade.
\begin{verbatim}
Výraz:	/Auto/g
Vyberie:	Auto
\end{verbatim}
Výraz nám teda nájde všetky 'Auto' v texte, ale čo keď chceme aj "auto" s malým písmenkom na začiatku?
V tomto prípade prichádza prvý koncept menom \emph{rozsah}. Zapisuje sa do hranatých zátvoriek.
Rozsahy môžeme aj negovať pomocou operátora \string^, aby sme vybrali všetky znaky okrem tých v rozsahu.

\begin{verbatim}
Výraz:	/[aA]uto/g
Vyberie:	auto Auto
\end{verbatim}
Teraz náš výraz vyberie 'auto' aj 'Auto' z textu, lebo sme v rozsahu pre prvé písmenko špecifikovaly \emph{malé} a aj \emph{veľké} a.
Ak by sme chceli vybrať všelijaké veľké a malé písmenko, tak rozsah zapíšeme s '-' ako interval.

\begin{verbatim}
Výraz: 	/[a-zA-z]uto/g
Vyberie: 	auto Auto xuto puto
\end{verbatim}
Tento výraz vyberie všetky prípady, kde je ľubovolné písmenko a za ním 'uto'.

\begin{verbatim}
Výraz: 	/[^a-zA-z]uto/g
Vyberie: 	0uto 1uto $uto !uto
\end{verbatim}
Výraz vyberie reťazce s lubovoľným znakon okrem písmen a s 'uto'.

\subsubsection{Kvantifikátory}
Začneme opäť s jednoduchím výrazom pre 'Auto'.
\begin{verbatim}
Výraz:	/Auto/g
Vyberie:	Auto
\end{verbatim}
Ako by sme vybrali opakujúce sa znaky, alebo písmena, ktoré tam môže a nemusia byť?
Máme na to následujúci špeciálne znaky určené pre kvantifikáciu. Tieto operátory sa aplikujú na jednotlivé znaky, ale môžu aj na rozsahy a skupiny.

\begin{table}
\centering
\begin{tabular}{|l|l|} 
\hline
\textbf{Operátor} & \textbf{Popis}                                  \\ 
\hline
+                 & musí byť aspoň raz, ale môže viac krát          \\ 
\hline
?                 & môže a nemusí tam byť                           \\ 
\hline
*                 & môže byť 0-krát alebo viac krát                 \\ 
\hline
\{min, max\}      & vyskytuje sa aspoň min-krát a najviac max-krát  \\
\hline
\end{tabular}
\caption{RegEx Kvantifikátory}
\end{table}

\begin{itemize}
\item
\begin{verbatim}
Výraz:	/A+uto/g
Vyberie:	AAuto Auto AAAAAAAAAAuto
\end{verbatim}
\item
\begin{verbatim}
Výraz:	/A?uto/g
Vyberie:	Auto uto
\end{verbatim}
\item
\begin{verbatim}
Výraz:	/A*uto/g
Vyberie:	AAuto Auto AAAAAAAAAAuto uto
\end{verbatim}
\item
\begin{verbatim}
Výraz:	/[Aa]{3,3}uto/g
Vyberie:	AAAuto AAauto Aaauto aaauto
\end{verbatim}
\end{itemize}

\subsection{Špeciálne znaky a skupiny znakov}
\begin{verbatim}
Výraz:	/Auto/g
Vyberie:	Auto
\end{verbatim}

Dokážeme už vybrať všetky vopred známe znaky a prípadne ich opakovanie. A čo keď nevieme aký znak bude prvý? Alebo ak sa nám nechce vypisovať každé písmenko abecedy a všetky známe znaky. Našťastie nemusíme, tvorcovia regulérnych výrazov na to mysleli.
Medzi takéto špeciálne znaky a skupiny znakov patrí.

\begin{table}[!ht]
	\centering
	 \begin{tabular}{|l|l|l|}
	\hline
		\textbf{Znak / Znaková skupina} & \textbf{Slovný popis} & \textbf{Rozsah} ~ \\ \hline
	        . & lubovolný znak okrem nového riadku '$\backslash$n' & ~ \\ \hline
        	\string^ & začiatok reťazca & ~ \\ \hline
        	\$ & koniec reťazca & ~ \\ \hline
	        $\backslash$s & všetky 'whitespace' znaky & [ $\backslash$f$\backslash$n$\backslash$r	$\backslash$v] \\ \hline
        	$\backslash$S & všetko okrem 'whitespace' znakov & [\string^ $\backslash$f$\backslash$n$\backslash$r$\backslash$t$\backslash$v] \\ \hline
	        $\backslash$w & všetky znaky slov & [a-zA-Z0-9\_] \\ \hline
	        $\backslash$W & všetko okrem znakov slov & [\string^a-zA-Z0-9\_] \\ \hline
        	$\backslash$d & všetky číslice & [0-9] \\ \hline
        	$\backslash$D & všetko okrem číslic & [\^0-9] \\ \hline
	\end{tabular}
	\caption{Špeciálne znaky a znakové skupiny}
\end{table}

\subsubsection{Skupiny}
Skupiny slúžia na segmentáciu nášho vyhladávania alebo pridávania podmienok.
Tieto skupiny sú nápomocné pri programovaní, kde môžeme k jednotlivým skupinám pristupovať podľa pomenovania.
Taktiež ich môžeme použiť na nájdenie reťazca, ktorý nechceme vo finále vybrať ako napríklad pri práci s CSV (Comma Seperated Values) súbormi, kde jednotlivé záznamy sú zakončené čiarkou, ale nechceme vybrať čiarku.

\begin{verbatim}
Vstup: 	 Name,Email,Phone Number,Address
Výraz:	/[^,\n]+(?=[,\n]?)/g
Vyberie:	'Name', 'Email', 'Phone Number', 'Address'
\end{verbatim}
Tento výraz vyberá záznamy z CSV zoznamu hodnôt a teda vyberá všetko pokiaľ nenájde čiarku, alebo nový riadok.
Je možné tento výraz vylepšiť aby ignoroval medzeri na začiatku zápisu, alebo aby mohla byť čiarka v zázname, ale to necháme ako príklad čitateľovi.

Skupinu dokážeme nastaviť tak, aby sme ju nezahŕňali do výsledkov. Taktiež ich dokážeme použiť ako podmienku a vyhodnotiť podľa toho či náš reťazec uložiť, alebo preskočil.

\begin{table}
\centering
\caption{RegEx Skupiny}
\begin{tabular}{|l|l|} 
\hline
\textbf{Zápis}     & \textbf{Popis}                                                 \\ 
\hline
(x)                & Zachytávacia skupina                                           \\ 
\hline
(?Názov\_Skupinyx) & Pomenovaná skupina                                             \\ 
\hline
(?:x)              & Skupina ktorú nezachytávame                                    \\ 
\hline
(?=x)              & Skupina, ktorá ak nie je za výrazom, tak záznam vynechávame    \\ 
\hline
(?!x)              & Skupina, ktorá ak je za výrazyom, tak záznam vynechávame       \\ 
\hline
(?=x)              & Skupina, ktorá ak nie je pred výrazom, tak záznam vynechávame  \\ 
\hline
(?!x)              & Skupina, ktorá ak je za výrazyom, tak záznam vynechávame       \\
\hline
\end{tabular}
\end{table}
\begin{verbatim}
Výraz:	/[Aa]{3,3}uto/g
Vyberie:	AAAuto AAauto Aaauto aaauto
\end{verbatim}

\subsection{Nahrádzanie}
Prebieha s pomocou nástrojov alebo programovacieho jazyku. Doteraz sme iba vyhľadávali skupiny reťazcov, teraz môžeme do programu zadať náš regulérny výraz a dosadzovací text. Program každú nájdenú skupinu nahradí za náš text a namiesto toho aby nám vrátil všetky nájdené skupiny, tak nám dá nový text po dosadení.

\begin{verbatim}
Vstup:			'I love cats, do you love cats as well?'
Výraz:			/cats/g
Dosadzovací text: 	dogs
Výsledok:			'I love dogs, do you love dogs as well?'
\end{verbatim}

\subsection{Modifikátory}
Doposial sme menili výraz medzi //, ale konečne sa dostávame mimo lomítka.
Tieto modifikátory dávajú informáciu programu vyhodnocujúcemu nášho regulérneho výrazu extra informácie.
Taktiež sa nazývajú vlajky a môžu ich byť vybraných viacero.

\begin{verbatim}
Výraz:	/auto/gi
Vyberie:	Auto auto AUto aUto
\end{verbatim}

\begin{table}[!ht]
    \centering
    \begin{tabular}{|l|l|l|}
    \hline
        \textbf{Možnosť} & \textbf{Popis} & \textbf{Znak} \\ \hline
        global & vyhľadáva všetky zhody namiesto iba prvej & g \\ \hline
        case insensitive & nerozlišuje medzi veľkými a malými písmenkami & i \\ \hline
        multiline & znak \string^ a \$ ukazujú na začiatok a koniec riadka & m \\ \hline
        single line & znak . bude vyberať aj nový riadok '$\backslash$n' & s \\ \hline
        unicode & pridáva možnosť používať & u \\ \hline
        sticky & vyraz začne vyhľadávať na konci minulého vyhľadávania & y \\ \hline
    \end{tabular}
    \caption{RegEx modifikátory}
\end{table}

\section{Aplikácia}
Posnažíme sa vylistovať niekoľko populárnych programov, ktoré pravdepodobne už poznáte a kde môžete uplatniť svoje nadobudnuté vedomosti.

Taktiež aj zopár bežne používaných programovacích jazykoch, ktoré majú podporu pre regulérne výrazy a ukážeme si príklad využiťia v programovaciom jazyku python.

\subsection{Textové editory}
Použitie regulérnych výrazov je veľmi situaćné pri bežnej práci s textovými editormi, ale občas sa stane že potrebujeme niečo premenovať, alebo vyhladať konkrétny reťazec a naše bežné vyhladávanie s konštatným textom nestačí. 

V týchto prípadoch môžeme využiť regulérne výrazy, vďaka ich užitočnosti sú pomerne často podporované.

Vylistujeme si zopár textových editorov čo podporujú ich funkcionalitu a aj úroveň podpory.

\begin{enumerate}
	\item Microsoft Visual Studio Code
	\begin{itemize}
		\item Populárny textový editor s kompatibilitou pre väčšinu súborových formáto
		\item Podporuje používanie regulérnych výrazov pri vyhľadávaní textu a nahrádzaní textu
	\end{itemize}
	\item MikText - TeXworks
	\begin{itemize}
		\item Integrované prostredie pre prácu s LaTeX dokumentami
		\item Podporuje používanie regulérnych výrazov pri vyhľadávaní textu a nahrádzaní textu
		\item Vrati celé riadky namiesto skupín
	\end{itemize}
	\item Microsoft Visual Studio
	\begin{itemize}
		\item Integrované vývojové prostriedie pre programovanie v množstve programovacích jazykoch
		\item Podporuje používanie regulérnych výrazov pri vyhľadávaní textu a nahrádzaní textu
	\end{itemize}
\end{enumerate}


\subsection{Programovacie jazyky}
V prípade skriptovacieho jazyka Perl si môžete prečítať papier \cite{Bioperl} v ktorom autor predstavuje skriptovacií jazyk Perl a taktiež obsahuje sekciu pre regulérne výrazy.

Okram skriptovacieho jazyka Perl, regulérne výrazy podpurujú aj následovné jazyky a pravdepodobne aj všetky jazyky, ktoré sa dnes bežne používajú v priemysle.

\begin{enumerate}
	\item Javascript
	\begin{itemize}
		\item Skriptovací jazyk s hlavným účelom pre webové stránky
		\item Obsahuje podporu pre takmer všetky funkcie regulérnych výrazov
		\item Funkcionalita môže závisieť od prehliadača v ktorom daný skript beží
	\end{itemize}
	\item Python
	\begin{itemize}
		\item Populárny skriptovací jazyk vhodný pre programy ktoré sú časovo nekritické
		\item Jednoduchý pre začiatočníkov
		\item Knižnica 're' poskytuje funkcionalitu regulérnych výrazov
	\end{itemize}
	\item C++
	\begin{itemize}
		\item Jazyk používaný v priemysle pre aplikácie, kde výkon je nutnosť
		\item Štandarná knižnica poskytuje implementáciu regulérnych výraz
		\item Definície funkcií sa nachádzajú v hlavičkovom súbore 'regex'
	\end{itemize}
\end{enumerate}

\subsection{Príklady použitia}

\subsubsection{Validácia vstupu používateľa}
Nápad je pomerne jednoduchý. Napíšeme výraz, ktorý následne aplikujeme na používateľov vstup. Ak nám vráti zhodu a tá zhoda je celý vstup, tak potom je vstup platný, inak je vstup nevhodný.

Platný email
\begin{verbatim}
Vstup:	'vsprintf@comcast.net' 'crobles@gmail.com' 'skola@is.stuba.sk' 'test@.com' 'st@gmail.com'
Výraz:	/^\w+@\w+[\w.]+$/gm
Vyberie:	'vsprintf@comcast.net' 'crobles@gmail.com' 'skola@is.stuba.sk'
\end{verbatim}

Telefónne číslo
Kontroluje iba bežné čísla, čiže 112 a iné špeciálne telefónne čísla vyradí.

\begin{verbatim}
Vstup:	'+421940264961' '+421908127668' '0950279858' '0911538050' '0911 538 050' '+420 757 205 203' '948840301'
Výraz:	/(?:\+\d{3,3}|0)(?: )?\d{3,3}(?: )?\d{3,3}(?: )?\d{3,3}/gm
Vyberie:	'+421940264961' '+421908127668' '0950279858' '0911538050' '0911 538 050' '+420 757 205 203'
\end{verbatim}

\subsubsection{Načítavanie údajov z dátového súboru}

Máme JSON súbor, v ktorom sú údaje z vozidiel MHD v Bratislave. Drží informácie ako identifikátor vozidla, liknu obsluhy, GPS pozíciu a čas aktualizácie záznamu deného vozidla. 

Chceme tieto údaje spracovať programovo a teda ich najprv musíme načítať. Keďže máme zoznam vozidiel, inými slovami pole a údaje sú pomenované, tak nám regulérne výrazy podstatne uľahčia život.

Nadovšetko musím dať na vedomie. Tento spôsob je pomalý a existujú lepšie spôsoby ako pracovať s JSON formátom a teda toto je iba príklad využitia.

\begin{verbatim}
Vstup:
[
  {
    "vehicleNumber": 1021,
    "lineNumber": 53,
    "gpsLatitude": 48.177199,
    "gpsLongitude": 17.166676,
    "lastModified": "2023-09-22T16:40:23.1371261+02:00"
  },
  {
    "vehicleNumber": 2339,
    "lineNumber": 95,
    "gpsLatitude": 48.190326,
    "gpsLongitude": 17.134353,
    "lastModified": "2022-12-22T10:34:01.96552"
  },
  {
    "vehicleNumber": 7406,
    "lineNumber": 4,
    "gpsLatitude": 48.142004,
    "gpsLongitude": 17.089938,
    "lastModified": "2023-09-22T16:40:22.8595525+02:00"
  }
]
Výraz: /(?<="vehicleNumber": )\d+/g
Výstup: '1021' '2339' '7406'
\end{verbatim}

V programovacom jazyku python by to vyzeralo následovne.
\begin{verbatim}
import re

vstup = """
[
  {
    "vehicleNumber": 1021,
    "lineNumber": 53,
    "gpsLatitude": 48.177199,
    "gpsLongitude": 17.166676,
    "lastModified": "2023-09-22T16:40:23.1371261+02:00"
  },
  {
    "vehicleNumber": 2339,
    "lineNumber": 95,
    "gpsLatitude": 48.190326,
    "gpsLongitude": 17.134353,
    "lastModified": "2022-12-22T10:34:01.96552"
  },
  {
    "vehicleNumber": 7406,
    "lineNumber": 4,
    "gpsLatitude": 48.142004,
    "gpsLongitude": 17.089938,
    "lastModified": "2023-09-22T16:40:22.8595525+02:00"
  }
]"""

regulerny_vyraz = re.compile('(?<="vehicleNumber": )\d+')
vystup = re.findall(regulerny_vyraz, vstup)

print(vystup) # ['1021', '2339', '7406']
\end{verbatim}

\section{Záver}
V tomto článku sme si prešli prakticou časťou regulérnych výrazov a nie teóriou ako sú implementované. Prešli sme ich zápis, teda syntax a význam špeciálných znakov.
Ukázali si pár ukážkových príkladov na pochopenie a vylistovali aj niekoľko prípadov, kde by sa dali využiť. Pevne verím že sme sa spoločne oboznámili s touto témou a dobre nám poslúži do budúcna pri práci ako informatici.

V prípade nenaplnenej túžby pochopenia implementácie, či záujmu ako to môže fungovať, doporučujem papier \cite{RegEx_Implementation} v ktorom autor má jasný diagram a popis fungovania regulérnych výrazov.

\clearpage
\bibliography{literatura}
\bibliographystyle{plain} % prípadne alpha, abbrv alebo hociktorý iný

% https://citeseerx.ist.psu.edu/pdf/d06c81bf356235854d83b6bfe29d00eb866926a8
% https://developer.mozilla.org/en-US/docs/Web/JavaScript/Guide/Regular_Expressions/Cheatsheet
% https://www.keycdn.com/support/regex-cheat-sheet
% https://www.codeguage.com/courses/regexp/flags
% https://www.jetbrains.com/help/go/tutorial-finding-and-replacing-text-using-regular-expressions.html

% https://regexr.com/
\end{document}